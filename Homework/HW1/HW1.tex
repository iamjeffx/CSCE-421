\documentclass[12pt]{article}

\usepackage{times}
\usepackage{amsmath}
\usepackage{latexsym}
\usepackage{fullpage}
\usepackage{graphicx}
\usepackage{amsfonts}

\graphicspath{ {./images/} }

\newcommand{\NOT}{\neg}
\newcommand{\AND}{\wedge}
\newcommand{\OR}{\vee}
\newcommand{\XOR}{\oplus}
\newcommand{\IMPLIES}{\rightarrow}
\newcommand{\IFF}{\leftrightarrow}
\newcommand{\E}{\exists}
\newcommand{\A}{\forall}

\setlength{\parskip}{.1in}

\renewcommand{\baselinestretch}{1.1}

\begin{document}

\begin{center}

{\bf
CSCE 421\\
HW 1\\
Jeffrey Xu\\
09/07/20\\
}

\end{center}

{\bf 1.1.} The gradient of a multi-variate function is a vector of the respective partials of the function. Our given function is shown below:

\begin{center}

$f(x,y)=x^{2}+ln(x)+xy+y^{3}$\\

\end{center}

The gradient vector is as shown below:

\begin{center}

$\begin{bmatrix}
f_{x}\\
f_{y}
\end{bmatrix}
$\\

\end{center}

The gradient values are shown below:

\begin{center}

$f_{x}=2x+x^{-1}+xy+y$\\
\bigskip
$f_{y}=x+3y^{2}$\\
\bigskip
$
\begin{bmatrix}
2x+x^{-1}+xy+y\\
x+3y^{2}
\end{bmatrix}
$
\end{center}

The gradient value for $(1,-1)$ can be computed by plugging those values into the gradient.

\begin{center}

$
\bigtriangledown f(1,-1)=
\begin{bmatrix}
2(1)+1^{-1}+1(-1)+(-1)\\
1+3(-1)^{2}
\end{bmatrix}=
\begin{bmatrix}
1\\
4
\end{bmatrix}
$

\end{center}

{\bf 1.2.} Given the definition of the gradient function, we need to calculate the gradient for the given function below:

\begin{center}

$f(x,y,z)=tanh(x^{3}y^{3})+sin(z)$\\

\end{center}

The partials of the function are shown below:

\begin{center}

$f_{x}=3y^{3}x^{2}sech^{2}(x^{3}y^{3})$\\
\bigskip
$f_{y}=3x^{3}y^{2}sech^{2}(x^{3}y^{3})$\\
\bigskip
$f_{z}=cos(z)$\\
\bigskip
$
\bigtriangledown f=
\begin{bmatrix}
3y^{3}x^{2}sech^{2}(x^{3}y^{3})\\
3x^{3}y^{2}sech^{2}(x^{3}y^{3})\\
cos(z)
\end{bmatrix}
$

\end{center}

The gradient value at $(-1,0,\pi/2)$ is shown below:

\begin{center}

$
\bigtriangledown f(-1,0,\pi/2)=
\begin{bmatrix}
(-1)^{2}(0)^{3}sech^{2}((-1)^{3}(0)^{3})\\
(-1)^{2}(0)^{3}sech^{2}((-1)^{3}(0)^{3})\\
cos(\pi/2)
\end{bmatrix}=
\begin{bmatrix}
0\\
0\\
0
\end{bmatrix}
$\\

\end{center}

{\bf 2.1.} The following matrix multiplication computation is shown below:

\begin{center}

$
\begin{bmatrix}
1 & -1 & 6 & 7\\
9 & 0 & 8 & 1\\
-8 & 5 & 2 & 3\\
10 & 4 & 0 & 1
\end{bmatrix}
\begin{bmatrix}
6 & 2\\
0 & -1\\
-3 & 0\\
11 & 4
\end{bmatrix}=$\\
\bigskip
$
\begin{bmatrix}
1(6)-1(0)-3(6)+7(11) & 1(2)-1(-1)+6(0)+7(4)\\
6(9)+0(0)-3(8)+11 & 9(2)-1(0)+8(0)+4\\
-8(6)+5(0)-3(2)+3(11) & -8(2)-5+2(0)+3(4)\\
10(6)+4(0)-3(0)+11 & 2(10)-4+4\\
\end{bmatrix}=
\begin{bmatrix}
65 & 31\\
41 & 22\\
-21 & -9\\
71 & 20
\end{bmatrix}
$\\

\end{center}

{\bf 2.2.} 
\begin{center}
$
\begin{bmatrix}
10\\
4\\
-1\\
8
\end{bmatrix}
\begin{bmatrix}
7 & 3 & 0 & 1\\
\end{bmatrix}=
\begin{bmatrix}
10(7) & 3(10) & 0(10) & 1(10)\\
4(7) & 3(4) & 0(4) & 1(4)\\
-7 & -3 & 0 & -1\\
8(7) & 3(8) & 0(8) & 1(8)\\
\end{bmatrix}
\begin{bmatrix}
70 & 30 & 0 & 10\\
28 & 12 & 0 & 4\\
-7 & -3 & 0 & -1\\
56 & 24 & 0 & 8
\end{bmatrix}
$

\end{center}

{\bf 2.3} 

\begin{center}
$
\begin{bmatrix}
9 & -3 & 1 & 6\\
\end{bmatrix}
\begin{bmatrix}
-3\\
4\\
-9\\
0\\
\end{bmatrix}=
9(-3)-3(4)+1(-9)+6(0)=-48
$
\end{center}

{\bf 3.1.} We want to calculate the $l_{0}$ norm for each vector given. The two vectors are shown below:

\begin{center}
$
{\bf a}=
\begin{bmatrix}
5\\
0\\
-1\\
4\\
\end{bmatrix}, {\bf b}=
\begin{bmatrix}

7\\
9\\
5\\
2\\
\end{bmatrix}
$\\
\end{center}

The $l_{0}$ norm counts the number of non-zero entries in the vector. For {\bf a}, we can see $l_{0}=3$ and for {\bf b}, $l_{0}=4$. 

{\bf 3.2.} The $l_{1}$ norm is the sum of the absolute values of each element in the vector:

\begin{center}

$l_{1}=\sum_{i=1}^{n}|a_{n}|$\\

\end{center}

For {\bf a}, $l_{1}=|5|+|0|+|-1|+|4|=10$ and for {\bf b}, $l_{1}=|7|+|9|+|5|+|2|=23$.

{\bf 3.3.} The $l_{2}$ norm is simply the square root of the sum of squares of each element:

\begin{center}

$l_{2}=\sqrt{\sum_{i=1}^{n}a_{i}^{2}}$\\

\end{center}

The values of the $l_{2}$ norm for each vector is shown below:

\begin{center}

$l_{2}({\bf a})=\sqrt{5^{2}+0^{2}+(-1)^{2}+4^{2}}=\sqrt{42}$\\
\bigskip
$l_{2}({\bf b})=\sqrt{7^{2}+9^{2}+5^{2}+2^{2}}=\sqrt{159}$\\

\end{center}

{\bf 3.4.} The $l_{\infty}$ norm computes the absolute maximal element in the vector.

\begin{center}

$l_{\infty}=max_{n}|x_{n}|$\\

\end{center}

For {\bf a}, $l_{\infty}=5$ and for {\bf b}, $l_{\infty}=9$. 

{\bf 4.1.} For rolling two dices where each has six faces numbered 1 to 6, the sample space is $\mathbb{Z}[1, 6]*\mathbb{Z}[1,6]$, or the cartesian product of the integers from 1 to 6 with itself. This produces pairs of integers $(1,1), (1,2), (1,3),...,(6,6)$. Note that this notation does specify that the first coordinate denotes the value of the first dice and the second coordinate denotes the value of the second dice. 

{\bf 4.2.} Our sample space has cardinality of 36. Therefore, the probability of any event $X$ can be calculated with the following equation.

\begin{center}

$\mathbb{P}(X)=\dfrac{|X|}{36}$\\

\end{center}

We need to find all the elements of the sample space that satisfy the property that the value of both dices add up to 10. We notice that $(5,5),(6,4),(4,6)$ satisfy this condition. Therefore $\mathbb{P}($Sum of dices equals 10$)=\dfrac{3}{36}=\dfrac{1}{12}$. 

{\bf 4.3.} This problem follows similar logic to the previous question. The elements of $\Omega$ that sum to 6 are $(1,5),(5,1),(2,4),(4,2),(3,3)$. There are 5 elements in this subset, therefore the probability of the sum equalling 6 is $\dfrac{5}{36}$. 

{\bf 5.1.} Recall that the expectation of a continuous random variable can be computed as shown below.

\begin{center}

$E[X]=\displaystyle \int_{-\infty}^{\infty}xp(x)dx$\\

\end{center}

We notice for the Uniform distribution, it is only defined for the interval $[a,b]$. The computation for the expectation of the Uniform distribution is shown below.

\begin{center}

$E[X]=\displaystyle \int_{a}^{b}\dfrac{x}{b-a}dx=\dfrac{1}{b-a}\displaystyle \int_{a}^{b}xdx=$\\
\bigskip
$\dfrac{1}{b-a}[\dfrac{1}{2}x^{2}]_{a}^{b}=\dfrac{1}{2(b-a)}[b^{2}-a^{2}]=\dfrac{(b-a)(b+a)}{2(b-a)}=\dfrac{b+a}{2}$\\

\end{center}

{\bf 5.2.} The standard deviation of a distribution is the square root of the variance of the distribution. The variance of a distribution can be computed using the following equation.

\begin{center}

$Var(X)=E[(X-E[X])^{2}]=E[X^{2}]-E[X]^{2}$\\

\end{center}

The only extra value we need to compute is $E[X^{2}]$. Note, $E[f(x)]=\displaystyle\int f(x)p(x)dx$.

\begin{center}

$E[X^{2}]=\dfrac{1}{b-a}\displaystyle\int_{a}^{b}x^{2}dx=\dfrac{1}{b-a}[\dfrac{1}{3}x^{3}]_{a}^{b}=\dfrac{1}{3(b-a)}[b^{3}-a^{3}]=\dfrac{(b-a)(a^{2}+ab+b^{2})}{3(b-a)}=$\\
\bigskip
$\dfrac{a^{2}+ab+b^{2}}{3}$\\
\bigskip
$Var(X)=\dfrac{a^{2}+ab+b^{2}}{3}-(\dfrac{b+a}{2})^{2}=\dfrac{a^{2}+ab+b^{2}}{3}-\dfrac{b^{2}+2ab+a^{2}}{4}=$\\
\bigskip
$\dfrac{4a^{2}+4ab+4b^{2}}{12}-\dfrac{3b^{2}+6ab+3a^{2}}{12}=\dfrac{b^{2}-2ab+a^{2}}{12}=$\\
\bigskip
$\dfrac{(b-a)^{2}}{12}$\\

\end{center}

Therefore the standard deviation is equal to $\dfrac{b-a}{\sqrt{12}}$. 

{\bf 6.1.} The computation for the accuracy of this detector is shown below.

\begin{center}

$A=\dfrac{37+55}{37+23+45+55}=0.575$\\

\end{center}

{\bf 6.2.} The balanced accuracy weighs each class equally to discourage skewed classification. Let $+1$ denote avocado and $-1$ denote no avocado.

\begin{center}

$A_{+1}=\dfrac{37}{37+23}=0.617$\\
\bigskip
$A_{-1}=\dfrac{55}{45+55}=0.55$\\
\bigskip
$A_{b}=\dfrac{A_{+1}+A_{-1}}{2}=\dfrac{0.617+0.55}{2}=0.5833$\\

\end{center}

{\bf 6.3.} The precision measures the number of true positives detected out of all positives detected.

\begin{center}

$P=\dfrac{TP}{TP+FP}=\dfrac{37}{37+45}=0.45122$\\

\end{center}

{\bf 6.4.} The recall measures the amount of true positives detected out of all positive values.

\begin{center}

$R=\dfrac{TP}{TP+FN}=\dfrac{37}{37+23}=0.617$\\

\end{center}

{\bf 6.5.} The equation for F-measure is shown below.

\begin{center}

$F_{1}=2\dfrac{PR}{P+R}=\dfrac{TP}{TP+0.5(FP+FN)}=\dfrac{37}{37+0.5(45+23)}=0.521$\\

\end{center}

{\bf 7.1.} 

\end{document}